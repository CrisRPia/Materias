\documentclass{article}

\begin{document}
    \begin{titlepage}
        \vspace*{\stretch{1.0}}
        \begin{center}
            \Large\textbf{Conjuntos}\\
            \large\textit{Cristian Rodríguez}
        \end{center}
        \vspace*{\stretch{2.0}}
    \end{titlepage}

    \section{Idea intuitiva de conjuntos}

    Un conjunto es una colección o agrupamiento de objetos distinguibles
    entre ellos que, generalmente, tienen algo en común.


    \section{Deifinición genérica}

    No importa el orden ni la repetición

    \[
        A = \{ 1, 2, 3 \} = \{ 1, 3, 2 \} = \{ 2, 3, 1, 2, 1 \}
    \]


    \section{Formas de definir conjuntos}

    \begin{enumerate}
        \item Por extensión: Nombre de los elementos. $A = \{ 0, 1, 2, 3, 4 \}$
        \item Por comprensión: Indicando qué propiedad cumplen:
        \[A = \{ x \in N \mid x \le 4\}\]
    \end{enumerate}


    \section{Igualdad de conjuntos}

    Dos conjuntos son iguales si tienen los mismos elementos.

    \subsection{Inclusión amplia}

    Decimos que un conjunto $A$ está incluido ampliamente en un conjunto
    $B$, si todo elemento de $A$ es también elemento de $B$.

    \[
        A \subseteq B \iff \forall x \in A \rightarrow x \in B
    \]

    Decimos que $A$ es un subconjunto de $B$.
    La inclusión amplia admite igualdad.

    \subsubsection{Propiedad reflexiva}

    Todo conjunto está incluido en sí mismo.

    \subsection{Inclusión estricta}

    \[
        A \subset B \iff \forall x \in A \rightarrow x \in B \land \exists x \in B \mid x \notin A
    \]

    \section{Conjunto vacío}

    Aquel conjunto que no tiene elementos lo llamamos conjunto vacío. \newline
    Símbolo: $\emptyset = \{ \}$. \newline
    No se debe usar ${ \{ \emptyset \} }$ \newline
    El conjunto vacío está incluido en todos los conjuntos.
    
    \section{Conjunto referencial o universal}

    Llamamos así al conjunto que contiene todos los elementos con los cuales
    estamos trabajando.

    Símbolo: $\Omega$

    \section{Conjunto de partes o conjunto potencia}

    Se llaman así al conjunto de todos los posibles subconjuntos de un
    conjunto.

\end{document}